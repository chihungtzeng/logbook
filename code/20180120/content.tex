至民國 104 年 5 月份為止,法務部律師查詢系統裡共可查詢到 14,714 位律
師 15 ,其中加入律師公會者則有 8,240 人,約佔 56\%,依據律師法第 11 條第 1
項規定:「律師非加入律師公會,不得執行職務」,因此可認實際得從事律師業
務者應為後者 8,240 人始為正確。

平均年齡 47.72

30 歲以下比例 (\%) 12.31

31-40 歲比例 (\%) 29.74

41-50 歲比例 (\%) 24.46

51-60 歲比例 (\%) 13.62

61-70 歲比例 (\%) 6.89

71 歲以上比例 (\%) 12.97

至於個人訴訟案件亦是以服務 10 件以下者為最多,然而也有一定數量的律師服務 11-20
件或 21-30 件,另外尚有律師服務 61 件以上;公司行號或機關團體訟訴案件同
樣也是以服務 10 件以下者為多,只有少部分律師服務 31 件以上;絕大多數律師
所處理的法扶案件也在 10 件以下,服務超過 20 件者樣本中每年皆少於 11 人。

接著透過表 4-10 呈現律師們在 101-103 年間執行之非訟案件數。在此,我
們考慮公司行號或機關團體非訟案件及個非訟案件兩類。我們發現,在有填答此
一問題的律師當中,每年皆有 200 位左右處理 0-10 件公司行號或機關團體非訟
案件,而有更多的律師每年僅處理 0-10 件個人非訟案件。此外,每年也有 30 位
以上的律師處理 31 件以上的公司行號或機關團體非訟案件。

大部分受僱於大型律師事務所律師的薪資,還
是與年資成正比,且大多數的律師年薪的起薪皆在 65 萬元以上。另外,若年資
達 5 年以上,大多數受僱律師年薪會在 165 萬元以上。
